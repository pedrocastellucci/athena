\documentclass{beamer}
%
% Choose how your presentation looks.
%
% For more themes, color themes and font themes, see:
% http://deic.uab.es/~iblanes/beamer_gallery/index_by_theme.html
%
\mode<presentation>
{
  \usetheme{default}      % or try Darmstadt, Madrid, Warsaw, ...
  \usecolortheme{default} % or try albatross, beaver, crane, ...
  \usefonttheme{default}  % or try serif, structurebold, ...
  \setbeamertemplate{navigation symbols}{}
  \setbeamertemplate{caption}[numbered]

} 

\setbeamertemplate{footline}[text line]{%
  \parbox{\linewidth}{\vspace*{0pt}
    \begin{flushright}Copyright \textcopyright\ 2016 \insertshortauthor.\end{flushright}}}

\usepackage[english]{babel}
\usepackage[utf8x]{inputenc}
\usepackage{color}

\linespread{1.3}
\definecolor{links}{HTML}{2A1B81}
\hypersetup{colorlinks,linkcolor=,urlcolor=links}

\title[Your Short Title]{An introduction to Julia programming}
\author{Pedro Belin Castellucci}
\date{November, 2016}


\begin{document}

\setbeamercolor{title}{fg=black}
\setbeamercolor{frametitle}{fg=black}


\begin{frame}
  \titlepage
\end{frame}


\section{Introduction}

\begin{frame}
  \begin{quote}\footnotesize

  ``We want a language that’s \textcolor<2->{red}{open source}, with a \textcolor<3->{red}{liberal license}. We want \textcolor<4->{red}{the speed of C} with the \textcolor<5->{red}{dynamism of Ruby}. We want a language that’s \textcolor<6->{red}{homoiconic}, with \textcolor<7->{red}{true macros like Lisp}, but with obvious, \textcolor<8->{red}{familiar mathematical notation like Matlab}. We want something \textcolor<9->{red}{as usable for general programming as Python}, \textcolor<10->{red}{as easy for statistics as R}, \textcolor<11->{red}{as natural for string processing as Perl}, \textcolor<12->{red}{as powerful for linear algebra as Matlab}, \textcolor<13->{red}{as good at gluing programs together as the shell}. Something that is \textcolor<14->{red}{dirt simple to learn}, yet keeps the most serious hackers happy. We want it \textcolor<15->{red}{interactive} and we want it \textcolor<16->{red}{compiled}.
\newline
\visible<17->{(Did we mention it should be as fast as C?)}''\\
  \href{http://julialang.org/blog/2012/02/why-we-created-julia}{Jeff Bezanson, Stefan Karpinski, Viral Shah, Alan Edelman (2012)}
     \end{quote}
\end{frame}


\begin{frame}{Who is using Julia?}
  \begin{itemize}\footnotesize
  \item[] Stanford University. Introduction to Multidisciplinary Design Optimization (Prof. Mykel J. Kochenderfer).
  \item[] MIT. Integer Programming and Combinatorial Optimization (Prof. Juan Pablo Vielma).
  \item[] MIT. Optimization Methods (Prof. Dimitris Bertsimas and Dr. Phebe Vayanos).
  \item[] University at Buffalo. Linear Programming (Prof. Changhyun Kwon).
  \item[] “Sapienza” University of Rome. Operations Research (Giampaolo Liuzzi).
    \item[] University of South Florida. Nonlinear Optimization and Game Theory (Prof. Changhyun Kwon).
  \end{itemize}
\end{frame}


\begin{frame}{What can I do with Julia?}
  \begin{itemize}\footnotesize
  \item[] Simple Audio IO in Julia (AudioIO).
  \item[] A neural network (BackpropNeuralNet).
  \item[] Support vector machines (LIBSVM, LIBLINEAR).
  \item[] Machine learning (MachineLearning).
  \item[] Bioinformatics and Computational Biology (Bio).
  \item[] Curve fitting (CurveFit).
  \item[] Describe and model financial markets (FinancialMarkets).
  \item[] Black-box optimization (BlackBoxOptim).
  \item[] Combinatorics (Combinatorics).
  \item[] Evolutionary and genetic algorithms (Evolutionary).
  \item[] Gurobi, GLPK, CPLEX, Cbc, Clp CoinOptServices, JuMP.
  \item[] \href{http://pkg.julialang.org/}{And more...}
  \end{itemize}
\end{frame}

\begin{frame}{What we will do?}
  \begin{itemize}
  \item[] Basic Julia programming.
  \item[] Explore JuMP for Mixed Integer Linear problems.
  \end{itemize}
\end{frame}

\begin{frame}
  \href{https://www.cs.utexas.edu/users/EWD/transcriptions/EWD08xx/EWD831.html}{Why numbering should start at zero}?
  \pause
  ``I don't know how many of you have ever met Dijkstra, but you probably know that arrogance in computer science is measured in nano-Dijkstras.'' (Alan Kay)
\end{frame}

\begin{frame}
  Implement a function with parameter $n$ to print the first $n$ prime numbers. 
\end{frame}

\begin{frame}
  Implement an algorithm to count the number of words in a file.
\end{frame}

\end{document}


%%% Local Variables:
%%% mode: latex
%%% TeX-master: t
%%% End:
